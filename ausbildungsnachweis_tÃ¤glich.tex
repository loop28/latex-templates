%
% README
% ======
%
% Umsetzung eines wöchentlichen Ausbildungsnachweises auf Basis der
% Ausbildungsnachweisvorlage der IHK Stuttgart in LaTeX.
%
% Dass der Nachweis bei zu viel Text nicht mehr auf eine Seite passt, ist als
% Feature zu verstehen und verhindert das Überladen des Nachweises.
%
% Diese Vorlage wurde mit PDFLaTeX getestet.
%
% LICENSE
% =======
%
% Copyright (c) 2019 Elias Mardaus <e.mardaus@runbox.com>
%
% Permission to use, copy, modify, and distribute this software for any
% purpose with or without fee is hereby granted, provided that the above
% copyright notice and this permission notice appear in all copies.
%
% THE SOFTWARE IS PROVIDED "AS IS" AND THE AUTHOR DISCLAIMS ALL WARRANTIES
% WITH REGARD TO THIS SOFTWARE INCLUDING ALL IMPLIED WARRANTIES OF
% MERCHANTABILITY AND FITNESS. IN NO EVENT SHALL THE AUTHOR BE LIABLE FOR
% ANY SPECIAL, DIRECT, INDIRECT, OR CONSEQUENTIAL DAMAGES OR ANY DAMAGES
% WHATSOEVER RESULTING FROM LOSS OF USE, DATA OR PROFITS, WHETHER IN AN
% ACTION OF CONTRACT, NEGLIGENCE OR OTHER TORTIOUS ACTION, ARISING OUT OF
% OR IN CONNECTION WITH THE USE OR PERFORMANCE OF THIS SOFTWARE.
%
\documentclass[10pt,a4paper]{article}
\pagestyle{empty}
\usepackage[utf8]{inputenc}
\usepackage{array}
\author{}
\date{}
%
% Die fortlaufende Nummer der Ausbildungsnachweise.
%
\title{Ausbildungs- und Tätigkeitsnachweis Nr.: X}
\begin{document}
\maketitle
%
% Daten zum Ausbildungsnachweis und dem Auszubildenden selbst.
%
\begin{tabular}{l|l}
  Name des Auszubildenden: & Vorname Nachname\\
  Ausbildungsjahr: & X\\
  Ausbildungswoche: & X\\
  Ausbildungswoche von - bis: & XX.XX.XXXX - XX.XX.XXXX\\
\end{tabular}
\\
\\\\
%
% Inhalt des Ausbildungsnachweises (Tag, Datum, Tätigkeiten).
%
\begin{tabular}{l l p{9cm} l}
  %
  % Tabellenüberschrift.
  %
  Tag: &  Datum: &  Tätigkeiten:\\
  \firsthline\\
  %
  % Bei Tätigkeiten außerhalb der Berufsschule "Vorlesung" und "Dozent durch
  % die jeweilig zutreffenden Orte und Personen erstetzen (z.B. "Betrieb" und
  % "Ausbilder").
  %
  % Montag.
  %
  Mo. & XX.XX.XXXX & Vorlesung / Dozent:\newline
    Vormittag: TODO.\newline
    Nachmittag: TODO.\\\\
  %
  % Dienstag.
  %
  Di. & XX.XX.XXXX & Vorlesung / Dozent:\newline
    Vormittag: TODO.\newline
    Nachmittag: TODO.\\\\
  %
  % Mittwoch.
  %
  Mi. & XX.XX.XXXX & Vorlesung / Dozent:\newline
    Vormittag: TODO.\newline
    Nachmittag: TODO.\\\\
  %
  % Donnerstag.
  %
  Do. & XX.XX.XXXX & Vorlesung / Dozent:\newline
    Vormittag: TODO.\newline
    Nachmittag: TODO.\\\\
  %
  % Freitag.
  %
  % (Falls nachmittägliche Tätigkeiten stattgefunden haben einfach wie
  % an den anderen Tagen eintragen und dabei die "newline" und "\\" Statements
  % nicht vergessen).
  %
  Fr. & XX.XX.XXXX & Vorlesung / Dozent:\newline
    Vormittag: TODO.\\
\end{tabular}
\\
\\\\
Durch die nachfolgende Unterschrift wird die Richtigkeit und Vollständigkeit der obigen Angaben bestätigt.
\\
\\\\
%
% Felder für die Unterschriften.
%
\begin{tabular}{l p{3cm} l l}
  &&\\ \cline{1-1} \cline{3-3}
  \scriptsize Datum, Unterschrift Auszubildender &&  \scriptsize Datum, Unterschrift Ausbilder\\
\end{tabular}
\end{document}
